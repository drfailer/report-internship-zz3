%------------------------------------------------------------------------------%
%                         discussion and possibilities                         %
%------------------------------------------------------------------------------%

\clearpage
\section{Discussion and prospect}

In ths serction we will discuss the results that we have obtained and the way we
have obtained them. We will also talk about the future of the project.

\subsection{FDS}

When we started working on \gls{fds}, we thought it was possible to port the
code to \gls{hh} manually. Some of the critical algorithms of the simulation
have been rewritten with our library, and we managed to have good results.
However, we eventually came to the conclusion that doing everything by hand was
not the best method. Firstly, this was a very difficult task considering the
size of the \gls{fds}'s code base and the large number of data that is used in
the program. Secondly, this method is not reproducible and cannot be applied on
other projects. Even if some tasks are technically feasible, it is always
important wonder if our methods can scale on bigger projects or can be easily
reused. One of the most important roles of a software engineer is the ability to
formulate the method he used to solve a problem, in order to automate it and
reuse it later on.\\

Currently, the simulation has not been optimized with \gls{hh} yet. However, it
may be in the future, when the refactoring tool will be operational.

\subsection{The serialization library}

The serialization library is a very interesting project with a unique design.
The tools employed in the library are not used in the other ones. The library
has currently a lot of features. However, despite the optimizations implemented,
it is still slower that the other libraries on the market. It is definitely
usable with \gls{hh}, but it would be interesting to improve the performance
even more.\\

Another potential issue for some developers is the heavy use of macros. Many
users dislike macros in C++ because they hide a lot of generated code.
Additionally, the preprocessor does not perform any verification when it
generates the code, which increase the probabilities of errors. Providing an
even simpler interface for using the library without the macros could be
beneficial.
