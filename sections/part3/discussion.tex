%------------------------------------------------------------------------------%
%                         discussion and possibilities                         %
%------------------------------------------------------------------------------%

\clearpage
\section{Discussion and prospect}

In ths serction we will discuss the results that we have obtained and the way we
have obtained them. We will also talk about the future of the project.

\subsection{FDS}

When we have started to work on \gls{fds}, we thought that it was possible to
port the code to \gls{hh} manually. Some of the critical algorithms of the
simulation have been rewritten with our library, and we managed to have good
results. However, we eventually came to the conclusion that doing everything by
hand was not the best method. Firstly, this was a very difficult task
considering the size of the \gls{fds}'s code base and the large number of data
that is used in the program. Secondly, this was not a reproducible method that
could be applied on other projects. Even if some tasks are technically feasible,
it is always important wonder if our methods can scale on bigger projects or can
be easily reused. One of the most important role of a software engineer is the
ability to formulate the method he has used to solve a basic problem, in order
to automate it and reuse it on other problems.\\

For now, the simulation has not been optimized with \gls{hh} yet. However, it
may be in the future, when the refactoring tool will be operational.

\subsection{The serialization library}

The serialization library is a very interesting project with a unique design.
The tools that we use in the library are not used in the other ones. The library
has currently a lot of features. However, despite the optimizations that have
been made, it remains slower that the other libraries on the market. This is an
issue that will need to be solved if we want to use it with \gls{hh}.\\

In the future, more optimizations will be added to accelerate the serialization
and reduce the size of the serialized data.
