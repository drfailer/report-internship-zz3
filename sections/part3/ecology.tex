%------------------------------------------------------------------------------%
%                                   ecology                                    %
%------------------------------------------------------------------------------%

\clearpage{}
\section{Ecology}

In this section, we will try to analyze the environmental impact of the
project.\\

The context of the project was the research on High Performance Computing. The
\gls{hpc} concept is not ecologic by itself. The purpose of this domain of
research is to ensure that applications utilise a maximum of the resources of
computers. These applications are usually executed on very energy-intensive
clusters. However, in some cases, \gls{hpc} can be used ecologically.

Firstly, when a lot of computing is required, \gls{hpc} research can be
important, not to reduce the energy consumption of clusters, but use the energy
efficiently. In fact, if no specific optimizations have been done on a computer,
as soon as the processor is powered on, it consumes energy, even if only half of
its cores are effectively used. Thanks to \gls{hpc} techniques, there are ways
to utilise this energy more efficiently, by using all the available resources to
accelerate the computation when it is possible.\\

Secondly, if \gls{hpc} itself is not directly linked to ecology, the techniques
developed in the domain can be used to improve environmental research. For
instance, simulation such as \gls{fds} is used a lot to predict environmental
events and help to find solutions.\\

Finally, a lot of researches are conducted to optimize the energy consumption of
the super-computers. Indeed, researchers have found out that it was possible to
control the power usage within processors. Interestingly, the fact of reducing
the power in a processor when it is not fully used can also have a positive
effect on the performance. \cite{song2009energy} presents some very interesting
work on power measurement and optimization. These techniques are used by the
super-computers member of the \textit{green500} list \cite{enwiki:1230059074}.
