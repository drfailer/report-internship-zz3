%------------------------------------------------------------------------------%
%                             state of the project                             %
%------------------------------------------------------------------------------%

\section{State of the project}

In this section, we will resume the work that has been done on \gls{fds} and the
serialization library, then we will explain what remains to be done.

\subsection{FDS}

The project of porting \gls{fds} to \gls{hh} is far from being done yet. We have
started by implementing some critical sections of the code such as the Cholesky
or the computation of the velocity. Then we have tried to analyse and reorganize
the code of the simulation. However, we eventually came to the conculsion that
there was too much work that shouldn't be done manually. The \gls{fds} project
has then been paused, and we have started to design a tool that we could use to
automate the generation of a \gls{hh}. The work that as been realized on the
simulation is important because it will help us to make choices in our future
work.\\

The tool that will automate the generation of the graphs hasn't been implemented
yet. In fact, we only are in the early stages of the design, however, the first
researches that have been made, and the first ideas that we have are promissing.
The problem is definitely not trivial, and it may take several months before we
are able to perform some basic transformations on the code.

\subsection{The serialization library}

The serialization library has been optimized compared to the previous version,
and some features have been added. There will be some more optimization that
will be added in the future since the serialization is still much slower
compared to the other libraries available on the market. For instance, the usage
of the RTTI is slow, and we may be able to use some liter meta-data on the types
like user defined enums. If we managed to get similar performances than the
other libraries, we will be able to use the library in \gls{hh}.
