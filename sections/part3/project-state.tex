%------------------------------------------------------------------------------%
%                             state of the project                             %
%------------------------------------------------------------------------------%

\section{State of the project}

In this section, we will resume the work that was done on \gls{fds} and the
serialization library, then we will explain what remains to be done.

\subsection{FDS}

The project of porting \gls{fds} to \gls{hh} is not done yet. We began by
implementing critical sections of the code such as the Cholesky decomposition or
velocity computaiton. Then we tried to analyse and reorganize the simulation
code. However, we eventually concluded that, due to the amout of work, this
process should not be done manually. The \gls{fds} project was then paused, and
we began the design of a tool that we could be used to automate the generation
of a \gls{hh} graph. The work completed on the simulation is important because
it will help us to make choices in our future work.\\

The tool that will automate the generation of the graphs has not been
implemented yet. We currently are in the early stages of the design. However,
the initial research and ideas we are promising. The problem is definitely not
trivial, and it may take several months before we can perform even basic
transformations on the code.

\subsection{The serialization library}

The serialization library has been optimized compared to the previous version
and now includes some additional features. There will be more optimization that
will be added in the future since the serialization is still slower compared to
the other libraries available on the market. Additionally, the new interface is
even simpler and more flexible, allowing the user to serialize classe by using
only one macro.
