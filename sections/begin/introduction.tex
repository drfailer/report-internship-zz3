%------------------------------------------------------------------------------%
%                                 introduction                                 %
%------------------------------------------------------------------------------%

\setcounter{page}{1}
\clearpage{}
\pagestyle{fancy}
\section*{Introduction}
\addcontentsline{toc}{section}{Introduction}

In this report, we will describe a project conducted at the National Institute
of Standards and Technology, as part of the third year internship at the
engineering school of ISIMA.\\

The purpose of the project was to optimize a fire simulation program, called
Fire Dynamic Simulator, using a C++ library named \gls{hh}, developed at the
NIST. The simulation code is currently able to scale on multiple nodes, however,
the performance within a single node could be improved. Our goal is to use
\gls{hh} and some \gls{hpc} techniques to accelerate the computation of the
simulation within a node.

Another objective will be to participate in the development of \gls{hh}. At
present, \gls{hh} does not support running on multiple nodes. However, some
researchers at the university of Utah are currently working on a port of
\gls{hh} to clusters. To achieve their goal, they will need a serialization
library in order to transfer the data on the network. Here, we plan to use a
serialization library originally developed as part of an ISIMA project. This
library will need to be optimized in order to be usable in \gls{hpc}
applications.\\

The report begins with an overview of \gls{hh} and its features. We then
describe the implementation of the Cholesky decomposition (used in \gls{fds})
using the library. This will serve as a first example on how \gls{hh} can be
used. In the third part, we will explain \gls{fds} and some of the work that has
been realised on the simulation's code. Following that, we will briefly
introduce a new project aimed at creating a tool that will assist porting
\gls{fds} to \gls{hh}. We then detail the modifications that have been brought
to the serialization library before concluding the report.

\clearpage{}
