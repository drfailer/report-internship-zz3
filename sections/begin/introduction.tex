%------------------------------------------------------------------------------%
%                                 introduction                                 %
%------------------------------------------------------------------------------%

\setcounter{page}{1}
\clearpage{}
\pagestyle{fancy}
\section*{Introduction}
\addcontentsline{toc}{section}{Introduction}

In this report, we will describe a project that has been realised at the
National Institute of Standards and Technology, as part of the third year
internship at the engineering school of ISIMA.\\

The purpose of the project was to optimize a fire simulation program, called
\gls{fds}, using a C++ library, called \gls{hh}, designed at the NIST. The
simulation code is currently able to scale on multiple nodes, however, the
performances within a node could be improved. Here we plan to use \gls{hh}
and some \gls{hpc} techniques to accelerate the computation of the simulation
within a node.

Another objective will be to participate in the development of \gls{hh}.
Currently, \gls{hh} is not able to run on multiple nodes. However, some
researchers of the university of Utah are currently working on a port of
\gls{hh} to clusters. To achieve their goal, they will need a serialization
library in order to transfer the data on the network. Here, we plan to use a
serialization library originally designed as part of an ISIMA's project. This
library will need to be optimized in order to be usable in \gls{hpc}
applications.\\

The report will start be a presentation of \gls{hh} and its features. Then we
will describe the implementation of the Cholesky decomposition (used in
\gls{fds}) using the library. This will serve as a first example on how \gls{hh}
can be used. In the third part, we will explain \gls{fds} and some of the work
that has been realised on the simulation's code. After that, we will briefly
introduce a new project that consists in the creation of a tool that will help
us in our effort of porting \gls{fds} to \gls{hh}. Then we will detail the
modifications that have been brought to the serialization library before
concluding the report.

\clearpage{}
