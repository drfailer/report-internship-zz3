%------------------------------------------------------------------------------%
%                                   abstract                                   %
%------------------------------------------------------------------------------%

\clearpage{}
\section*{Abstract}
% \addcontentsline{toc}{section}{Abstract}

This report describes the work conducted at the NIST, on a project that
consisted in the optimization of a fire simulation. The objective was to improve
the computing performance of \gls{fds}, on single computing node, using a C++
library called \gls{hh}. This library uses data-flow graphs to represent
parallel algorithms. The report also describes the improvements that have been
made to a serialization library, with the objective of utilizing it to transfer
data on the network on the cluster version of \gls{hh}.\\

Some simple algorithms such as the Cholesky's decomposition have been
implemented to demonstrate the capabilities of \gls{hh}. On this algorithm, we
managed to have slightly better performances than the openblas implementation.
Some parts of the simulation have also been optimized. Eventually, we came to
the conclusion that parallelizing \gls{fds} by hand was too difficult, and we
decided to implement a tool that could automate a part of the process. This tool
is briefly introduced in this report, however, it is not implemented yet.

For the serialization library, some new features have been added. However,
dispite the fact that some parts of the library have been optimized, the current
performances are still far below the best serialization libraries on the
market. The library still proposes a unique design, and more optimizations will
be made in the future.\\

Keywords: \textbf{C++}, \textbf{HPC}, \textbf{Hedgehog}, \textbf{FDS},
\textbf{simulation}, \textbf{serialization}.

\section*{Résumé}
% \addcontentsline{toc}{section}{Résumé}

Ce rapport décrit le travail, réalisé au NIST, sur un projet consistant en
l'optimisation d'une simulation de feu. L'objectif était d'améliorer les
performances de calcul de \gls{fds}, sur un unique nœud de calcul, en utilisant
une bibliothèque nommée \gls{hh}. Cette bibliothèque utilise des data-flow
graphs pour modéliser des algorithmes parallèles. Ce rapport décrit aussi les
améliorations qui ont été apportées à une bibliothèque de sérialisation, avec
pour objectif de l'utiliser pour transférer des données sur la version cluster
de \gls{hh} (version utilisée sur un cluster de calcul).\\

Des algorithmes simples tels que la décomposition de Cholesky ont été
implémentés pour démontrer les capacités de \gls{hh}. Sur cet algorithme, nous
avons réussi à obtenir de meilleures performances que l'implémentation de
openblas. Des parties de la simulation ont été optimisées. Finalement, nous
avons conclu que paralléliser \gls{fds} à la main été trop difficile. Nous avons
donc décidé d'implémenter un outil capable d'automatiser en partie cette tâche.
Cet outil est brièvement décrit dans ce rapport, cependant, il n'est pas encore
implémenté.

Quant à la bibliothèque de sérialisation, de nouvelles fonctionnalités ont été
ajoutées. Cependant, bien que certaines parties de la bibliothèque aient été
optimisées, les performances actuelles sont encore bien en dessous de celles des
meilleures bibliothèques du marché. La bibliothèque propose tout de même des
choix de conception unique, et plus d'optimisations seront apportées dans le
futur.\\

Mots-clés: \textbf{C++}, \textbf{HPC}, \textbf{Hedgehog}, \textbf{FDS},
\textbf{simulation}, \textbf{serialisation}.
