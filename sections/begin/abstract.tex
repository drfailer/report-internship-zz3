%------------------------------------------------------------------------------%
%                                   abstract                                   %
%------------------------------------------------------------------------------%

\section*{Abstract}
% \addcontentsline{toc}{section}{Abstract}

This report describes the research performed at NIST, for optimizing a fire
simulation. The objective was to improve the computing performance of \gls{fds},
on single computing node, using a C++ library called \gls{hh}. This library
represents parallel algorithms as data-flow graphs. The report also describes the
improvements that have been made to a serialization library, with the objective
of utilizing it to transfer data on the network on the cluster version of
\gls{hh}.\\

\gls{fds} is composed, amongst other algorithms, over Cholesky decomposition.
A version using \gls{hh} was implemented to demonstrate the library's
capabilities. We eventually managed to have a slightly better performance than
the OpenBLAS implementation. Some parts of the fire simulation have also been
optimized. Eventually, we came to the conclusion that parallelizing \gls{fds} by
hand was too difficult, and we decided to implement a tool that could automate a
part of the process. This tool is briefly introduced in this report, however, it
is not implemented yet.

For the serialization library, some new features have been added. However,
despite the fact that some parts of the library have been optimized, the current
performance are still far below the best serialization libraries available. The
library still proposes a unique design, and more optimizations will be made in
the future.\\

Keywords: \textbf{C++}, \textbf{HPC}, \textbf{Hedgehog}, \textbf{FDS},
\textbf{simulation}, \textbf{serialization}.

\section*{Résumé}
% \addcontentsline{toc}{section}{Résumé}

Ce rapport décrit le travail, réalisé au NIST, sur un projet consistant en
l'optimisation d'une simulation de feu. L'objectif était d'améliorer les
performance de calcul de \gls{fds}, sur un unique nœud de calcul, en utilisant
une bibliothèque nommée \gls{hh}. Cette bibliothèque utilise des data-flow
graphs pour modéliser des algorithmes parallèles. Ce rapport décrit aussi les
améliorations qui ont été apportées à une bibliothèque de sérialisation, avec
pour objectif de l'utiliser pour transférer des données sur la version cluster
de \gls{hh} (version utilisée sur un cluster de calcul).\\

\gls{fds} utilise la décomposition de Cholesky. Une version utilisant \gls{hh} a
été implémentée pour démontrer les capacités de la bibliothèque. Nous avons
finalement réussi à obtenir de meilleures performance que l'implémentation de
OpenBLAS. Des parties de la simulation ont aussi été optimisées. Finalement,
nous avons conclu que paralléliser \gls{fds} à la main été trop difficile. Nous
avons donc décidé d'implémenter un outil capable d'automatiser en partie cette
tâche. Cet outil est brièvement décrit dans ce rapport, cependant, il n'est pas
encore implémenté.

Quant à la bibliothèque de sérialisation, de nouvelles fonctionnalités ont été
ajoutées. Cependant, bien que certaines parties de la bibliothèque aient été
optimisées, les performances actuelles sont encore bien en dessous de celles des
meilleures bibliothèques du marché. La bibliothèque propose tout de même des
choix de conception unique, et plus d'optimisations seront apportées dans le
futur.\\

Mots-clés: \textbf{C++}, \textbf{HPC}, \textbf{Hedgehog}, \textbf{FDS},
\textbf{simulation}, \textbf{serialisation}.
