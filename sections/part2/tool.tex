%------------------------------------------------------------------------------%
%                                     tool                                     %
%------------------------------------------------------------------------------%

\clearpage{}
\section{Automatic graph generation}

In Section \ref{sec:fdsconcl}, we discussed that porting \gls{fds} to
\gls{hh} was a very difficult task that could not be done manually. Furthermore,
this was not a method that could be easily used on even bigger project.
Fortunately, while parallelizing different algorithms with \gls{hh}, we observed
some common patterns. For instance, the operation of splitting the code into
simple functions, that can be bound to tasks in the graph, is done multiple
times. Another common pattern is analysing the data used by a function or an
algorithm, to determine the form of the data that will flow into the graph.
Based on these observations we wondered if it would be possible to create a tool
that would be able to perform these operations automatically. To answer this
question, we have initiated a project in collaboration with \textit{Semantic
Designs}, a company that has a lot of experience with the technologies we plan
to use for our tool. In this section, we will briefly describe the
specifications of the tools as well as some ideas that we currently have for its
implementation. We will not go further into the details as we are still in the
early stages of the project.

\subsection{The specifications}

The purpose of the tool is to translate the code into C++ and then refactor it
and build a graph. There are two reasons for translating the code into C++
before modifying it. Firstly, this approach will enable us to work with any
language. Indeed, transformations on the code can vary depending on the
language, so it will be easier to work only with C++. Secondly, there is a
technical limitation of the compilers imposed by the C standard. In fact,
\gls{fds} manipulates many data, and some of the subroutines use a lot of
variables. Currently, the variables are global, however, managing the memory in
C++ will require to pass these variables as parameters. According to the C
standard \cite{cstd}, there can only be a maximum of 127 parameters in a
function, which could be a limitation in our case\footnote{This is the C
standard and not the C++ one since the functions that are bind to the Fortran
subroutines are \texttt{extern "C"}.}. The only way to pass variables to Fortran
subroutines is by using pointers on fundamental types. It is not possible to
bind a Fortran type to a C structure that contains pointers
\cite{fortranbindstruct}. Because of this limitation, if a subroutine needs to
access more than 127 elements that are not fundamental (arrays for instance), it
will not be possible to call this subroutine from C++.

After translating the source code into C++, the tool must be capable to make
some changes to the code. For instance, it should remove all global variables
automatically. This task is complex because there are a lot of things consider
(is a variable used elsewhere, should it be local anyway\footnote{In FDS, loops
variables can be global.}, \dots). Another operation is reorganizing the code
into functions. Most importantly, this reorganization must be done in a way that
allows parallelism. Additionally, after the modifications, the tool should
generate the \gls{hh} graph automatically. This involves even more complexity
because besides of generating the tasks, we also have to generate the types of
the data that will flow into the graph.

The tool should also have some additional secondary features. We plan to
introduce special comments to the code, using a syntax similar to doxygen, to
guide the tool on how to reorganize the code. For instance, we might use a
\textit{task} or a \textit{graph} comment, to indicate that a certain section of
the code should belong to a task or a sub-graph. Finally, we intend to create a
custom language for automating some operations on the code (extract a function
for example). The objective is to train a \gls{llm} to generate these
operations.

Now that we have defined the specifications of the tool, shall we talk about the
implementation.

\subsection{Implementation}

Implementing this tool will require a transpiler to parse the Fortran code and
translate it in C++. We will then modify the C++ \gls{ast} of the program in
order to perform the operations that we described earlier. The code will be
transformed in a manner similar to that used by compilers, however, the goal is
not to optimize the code but to generate a \gls{hh} graph. Fortunately,
\textit{Semantic Designs} already have some libraries and tools that can be used
for transpiling the code and change the \gls{ast}. However, their \gls{parser}
will be adapted in order to support the new comment syntax.\\

This tool will be a very interesting and useful project that will utilize the
compilers technologies in an innovative way. Initially, we will focus on
generating code for \gls{hh}, but other directions may be explored in the
future.
