%------------------------------------------------------------------------------%
%                                   hedgehog                                   %
%------------------------------------------------------------------------------%

\section{Hedgehog}
\label{sec:hh}

\gls{hh} is a template library that uses data-flow graphs to create
parallel algorithms \cite{bardakoff2021analysis}. The library is the direct
descendent of \gls{hmbe} that also uses data-flow graphs
\cite{blattner2017model}.

In this section, we will present some of the features of the library and how
we can use it to express an algorithm under the form of a graph.

\subsection{The data-flow graphs}

As we have explained in the introduction of this section, we need to express our
algorithms under the form of data-flow graphs. A data-flow graph is a graph
where edges represent data streams and vertices represent tasks
\cite{kavi1986formal} and \cite{bardakoff2021analysis}.

% TODO figure

In each task, the inputted data will be transformed into new data that will be
the input of other tasks. In \gls{hh}, the data are transmitted between nodes
using shared pointers in order to be as efficient as possible even with big
data types. % todo: reformulate

\subsection{The nodes}

To design an algorithm using Hedgehog, we have to create a graph in which are
data will flow. A graph is composed of nodes of different kinds that we will
describe in this section.

\subsubsection{Tasks}

The first type of nodes are the tasks. These nodes are made to be duplicated and
ran in different threads. As all other nodes, a task can have multiple input and
output types.

\subsubsection{States}

The states are special nodes that are only made for synchronisation and cannot
be duplicated. They are thread safe and can be used to control the data-flow in
the graph. We have to be careful when creating a state as it locks the program
using a mutex. This means that only simple operations have to be made in the
states in order not to slow the program execution.

% example

\subsubsection{State managers}

One of the most interesting features of \gls{hh} is the fact that it allows the
creating of cycles in the graphs. This is particularly useful when we want to
repeat an operation multiple times. However, it is not possible to automatically
detect when we can break the cycle at runtime. This has to be specified by the
user using a state manager.

% todo

% example

\subsubsection{Graphs}

The graph itself is a node as well which means that we can use a graph in
another one. In a graph, we can create edges between different nodes in order
form them to be able to communicate.

% example

\subsection{Example}
