%------------------------------------------------------------------------------%
%                                  objectives                                  %
%------------------------------------------------------------------------------%

\clearpage
\section{Objectives}

The objective of the internship is to study the usage of \gls{hh} and the
data-flow graphs in the context of \gls{hpc} applications. One of the objectives
of the project was to use \gls{hh} to accelerate a simulation application called
\gls{fds}.

\gls{hh} is a \gls{hpc} library written in C++ and created at the NIST. It uses
data-flow graphs to express algorithms composed of tasks that can be run
asynchronously. This library will be described in the section \ref{sec:hh}.

\gls{fds} is a fire simulation also created at the NIST. The principle is to
simulate fire evolution for a certain configuration. The configuration is given
threw a file that will contain information such as the size of the room, the
walls, the windows and doors, the material on fire at the beginning, \dots T

Currently, \gls{fds} is written in Fortran (standard 2018) and it is parallelized
with MPI and OpenMP.

Another goal of the internship is to contribute to the development of \gls{hh}.
Currently, the library is not able to run on multiple nodes on clusters. The
current cluster version of \gls{hh} uses MPI to handle communication between the
nodes. However, this requires to serialize the data before the communication. To
solve this problem, it has been proposed to use a serialization library written
at ISIMA during the third year project. The objective will be to optimize the
serialization time and extend the features of the library in order to be able to
use it in \gls{hh}.
