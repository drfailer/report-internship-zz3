%------------------------------------------------------------------------------%
%                               problem analisys                               %
%------------------------------------------------------------------------------%

\clearpage{}
\section{Problem analysis}

The first step is to understand how to use \gls{hh}. To do that, the easiest way
to do this is by following the tutorials and creating initial programs to
explore the various features of the library.\\

The next step will be to study \gls{fds}'s source code and isolate the different
parts that will be parallelized using \gls{hh}. \gls{fds} is written in Fortran
and some part of the code are already parallelized with MPI or OpenMP.
Originally, the objective was not to rewrite the entire application in C++. We
aimed to split the simulation code into functions that could be used in C++.
Only a few critical algorithms should have been entirely rewritten in C++ in
order to optimize the performance. All the preliminary tests were conducted
with this approach in mind, however, we will see that we eventually concluded
that this approach was not feasible because of the technical limitations of the
compilers\footnote{These limitations are imposed by the standard and not the
compilers themselves.}. Furthermore, when the project has been started, our goal
was to port \gls{fds} to \gls{hh} manually. Once again, after some first tests,
we finally decided to automate the process. To do that, our goal is to create a
refactoring tool that will perform \gls{ast} transformations and that will
generate the code of the \gls{hh} graph. This report will briefly describe the
specifications of this tool, but it will not detail its development since this
project has just begun. It is good to mention that this could not have been
predicted without a prior knowledge of the simulation's code, and the first
tests that have been made have played an important role in the definition of the
specifications of the tool.\\

Concerning the serialization library, the primary optimization idea was to use
binary serialization instead of a human-readable format. Indeed, the library
originally used a JSON like format, however, the binary one is lighter and does
not need the usage of a \gls{parser}. Furthermore, another optimization will be
to use strings instead of streams. Indeed, as we will explain, the streams are
usually implemented with a linked list. They require to frequently access the
RAM of the computer whereas strings are arrays that can be easily loaded in the
CPU cache.\\

Some features will also be added to the library in order to make it more
powerful. The first version like the support for static and dynamic arrays.
Furthermore, some data structure can be difficult to serialize because of
pointers. A workaround to this issue is to give the possibility to the user to
execute code during the serialization. Finally, the old version used some macros
that were generating a lot of code and that was very difficult to maintain.
These macros need be removed in order for the library to be more usable.
