%------------------------------------------------------------------------------%
%                               problem analisys                               %
%------------------------------------------------------------------------------%

\clearpage{}
\section{Problem analysis}

The first thing that will need to be done is understanding how to use \gls{hh}.
To do that, the easiest way is to follow the tutorials and create first programs
to explore the various features of the library.\\

The next step will be to study \gls{fds}'s source code and isolate the different
parts that will be parallelized with \gls{hh}. \gls{fds} is written in Fortran
and some part of the code are already parallelized with either MPI or OpenMP.
Originally, the objective was not to rewrite the entire application in C++. We
wanted to keep the original code and split it into different libraries that
would have been used in the C++ code. Only a few critical algorithms should have
been entirely rewritten in C++ in order to optimize the performances. All the
preliminary tests have been made with this idea in mind, however, we will see
that we eventually came to the conclusion that this could not be possible
because of the technical limitations of the compilers\footnote{These limitations
are imposed by the standard and not the compilers themeselves.}. Furthermore,
when the project has been started, our goal was to port \gls{fds} to \gls{hh}
manually. Once again, after some first tests, we finally decided to automate the
process. To do that, our goal is to create a refactoring tool that will perform
\gls{ast} transformations and that will generate the code of the \gls{hh} graph.
This report will briefly describe the specifications of this tool, however, it
will not detail its creation since this project has just been started. It is
good to mention that this could not have been predicted without a prior
knowledge of the simulation's code, and the first tests that have been made have
played an important role in the definition of the specifications of the tool.\\

Concerning the serialization library, the primary idea to optimize it was to use
binary serialization instead of a human-readable format. Indeed, the library
originally used a JSON like format, however, the binary one is liter and does
not need the usage of a \gls{parser}. Furthermore, another optimization will be
to use strings instead of streams. Indeed, as we will explain, the streams are
usually implemented with a linked list. They require to frequently access the
RAM of the computer wherease the strings are arrays that can be easily loaded in
the CPU cache.\\

Some features will also be added to the library in order to make it more
powerfull. The first version laked the support for static and dynamic arrays.
Furthermore, some data structure can be difficult to serialize because of
pointers. A workarround to this issue is to give the possiblity to the user to
execute code during the serialization, so new serializer will accept functions.
Finally, the old version used some macros that was generating a lot of code and
that was very difficult to maintain. This macro has to be removed in order for
the library to be usable.
